\documentclass[uplatex]{jsarticle}

\usepackage{mylatex}
\usepackage{ap3}
\usepackage{ascmac}

\title{応用プログラミングⅢ [Q2] 中間課題}
\author{c0115114 菅野 路哉}
\date{2016年06月19日}

\usepackage[dvipdfmx]{graphicx}
\begin{document}
\maketitle
% --- main content ---

\section{目的}
古いSingleton実装の問題点を知り、問題点を補ったSingleton実装を行えるようにする。また、何故そのような問題が起きるのかを考察する。


\section{課題設定}
今回のレポートでは、課題2.1を選択する。古いSingleton実装であるOldSingleton.javaでは、Singletonのインスタンスが複数生成されてしまう場合が存在する。そのため、何故そのような事が起きるのか、それを防ぐにはどうしたら良いのかを考察する。考察には、OldMain.javaの実行結果と、OldSingleton.javaに修正を加えたNewSingleton.javaを用いたNewMain.javaの実行結果を用いる。

また、今回使用するOldSingleton.javaと、OldMain.javaは既に用意されているものを用いる。


\section{ソースコード}
\subsection{OldSingleton}
\lstinputlisting[
  caption=OldSingleton.java,
  label=src1,
  language=Java
]{./kg07/OldSingleton.java}

\subsection{OldMain}
\lstinputlisting[
  caption=OldMain.java,
  label=src2,
  language=Java
]{./kg07/OldMain.java}

\subsection{NewSingleton}
\lstinputlisting[
  caption=NewSingleton.java,
  label=src3,
  language=Java
]{./kg07/NewSingleton.java}

\newpage
\subsection{NewMain}
\lstinputlisting[
  caption=NewMain.java,
  label=src4,
  language=Java
]{./kg07/NewMain.java}

\section{実行結果}
\lstinputlisting[
  caption=OldMain.javaの実行結果,
  label=result,
  numbers=none
]{./kg07/old.result}

\lstinputlisting[
  caption=NewMain.javaの実行結果,
  label=result2,
  numbers=none
]{./kg07/new.result}

\section{解説}
挙動説明を簡易化するために、Singletonのインスタンス生成時にかかる時間を延ばす。そのため、コンストラクタ内で一秒停止している。また、インスタンスが生成された場合、「インスタンスを作成しました」のメッセージが出るようにしている。

\pgref{src1}では、古いSingleton実装が定義されている。
インスタンスの取得はgetInstanceメソッドを用いる。
getInstanceでは、自身のフィールドであるsingletonがnullだった場合、新しくOldSingletonインスタンスを生成し、singletonに代入する。
また、そのsingletonを戻り値とする。


\pgref{src2}では、古いSingletonを用いて、マルチスレッドによるSingletonのインスタンス参照を行う。
2つのサブスレッドを作成して、順次スレッドを実行する。
スレッド内では、OldSingletonのgetInstanceを呼び出し、取得したインスタンスのObjectIDを確認する。
それぞれのスレッドで異なったObjectIDが検出された場合、同一のインスタンスが取得出来ていない事が分かる。


\pgref{src3}では、改良したSingleton実装を定義した。
インスタンスの取得は、OldSingletonの場合と同様のgetInstanceメソッドを用いるが、修飾子にsynchronizedを付与して、並列実行が行われないようにする。
他の実装は、OldSingletonと変わらない。


\pgref{src4}では、OldSingletonの代わりにNewSingletonを用いて、OldMainと同様の処理を行う。


\section{考察・まとめ}
\pgref{result}では、「インスタンスを生成しました」のメッセージが2度出ている。
また、ObjectIDがそれぞれ異なっている。
以上の事から、それぞれのスレッドで参照しているSingletonインスタンスは別物である事が分かる。
これは、それぞれのスレッドから、getInstanceメソッドを同時に呼び出している事が原因である。
getInstanceメソッドでは、既存のインスタンスが作成されていない場合、インスタンスを生成する。
この時、インスタンス生成途中でgetInstanceメソッドが呼び出されると、既存のインスタンスは未だ無いため、追加でインスタンスを生成し始める。
そのため、OldSingletonでは複数のインスタンスが存在することになる。

\pgref{result2}では、「インスタンスを生成しました」のメッセージが1度出ている。
また、ObjectIDがそれぞれ一致している。
以上の事から、それぞれのスレッドで参照しているSingletonインスタンスは同一であることが分かる。

NewSingletonでは、OldSingletonで複数のインスタンスが生成される問題に対して、synchronized修飾子をgetInstanceメソッドに付与することで対策している。
synchronized修飾子は、そのブロックに対して処理が始まった際にロックを行い、別の場所からのアクセス要求があった場合、現在行っている処理が完了するまで待機させる。
そのため、今回の場合、インスタンス生成時にgetInstanceメソッドが呼び出されることが無くなり、インスタンスが1つに定まる。

% --- main content ---
\end{document}

