\documentclass[uplatex]{jsarticle}

\usepackage{mylatex}
\usepackage{ap3}
\usepackage{ascmac}

\title{応用プログラミングⅢ [Q2] レポート}
\author{C0115114 菅野路哉}
\date{2016年5月7月}

\usepackage[dvipdfmx]{graphicx}
\begin{document}
\maketitle
% --- main content ---

\section{第3回 課題2}
\pgref{kadai}のようなファイルから{\tt Iterator}パターンを使ってデータを1つ1つ読み取り、出力するプログラムを作成する。

ファイルから学籍番号を取り出すクラスは{\tt StudentDirectory}, そのイテレータに相当するクラスは{\tt StudentDirectoryIterator}という名前にする。
\lstinputlisting[
  caption=入力データが記録されたファイル,
  label=kadai,
  numbers=none
]{./en03/data.txt}


\section{ソースコード}
\subsection*{{\tt Aggregate}実装}
\lstinputlisting[
  caption=StudentDirectory.java,
  label=src1,
  language=Java
]{./en03/StudentDirectory.java}

\subsection*{{\tt Iterator}実装}
\lstinputlisting[
  caption=StudentDirectoryIterator.java,
  label=src2,
  language=Java
]{./en03/StudentDirectoryIterator.java}

\subsection*{実行ファイル}
\lstinputlisting[
  caption=Main.java,
  label=src3,
  language=java
]{./en03/Main.java}

\section{実行結果}
\lstinputlisting[
  caption=実行結果,
  label=result,
  numbers=none
]{./en03/result.txt}


\section{解説}
\pgref{src1}では、{\tt Iterable interface}に対して実装を行った。
{\tt Iterable}は{\tt Aggregate}に相当する。
コンストラクターでファイル名を受け取り、そのファイルを一行ずつ読み込む。
読み込んだデータを、{\tt List}である{\tt students}に記録していく。
データ件数は{\tt length}に記録する。


\pgref{src2}では、{\tt Iterator interface}に対して実装を行った。
{\tt hasNext}では、データ件数である{\tt StudentDirectory}のデータ件数と現在のインデックスを比較し、データが残っているかを判別する。
{\tt next}では、{\tt StudentDirectory}から{\tt getStudentAt}を使い、引数番目のデータを受け取る。

% --- main content ---
\end{document}

