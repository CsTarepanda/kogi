\documentclass[uplatex]{jsarticle}

\usepackage{amsmath}

\title{情報セキュリティーRSA暗号}
\author{c0<gakuseki> 名前}
\date{2016年06月18日}

\allowdisplaybreaks
\usepackage[dvipdfmx]{graphicx}
\begin{document}
\maketitle
% --- main content ---

\section{平文}
自分の学籍番号を用いて平文を求める。

初めに、十分に大きい素数であるpとqを自由に決定する。
今回、pとqは19, 31を用いる。

次に、pとqを掛けた値をnとし、学籍番号(<gakuseki>)をn - 2で割った余りに2を足したものを平文とする。また、nは公開鍵の1つである。
\begin{align*}
  p&=19\\
  q&=31\\
  n&=p \times q = 19 \times 31 = <n>\\
  平文&=(115114 \bmod (n - 2)) + 2 = (<gakuseki> \bmod <n>) + 2 = <m>
\end{align*}
n: <n>\\
平文: <m>


\section{秘密鍵生成}
p - 1とq - 1の最小公倍数を求める。求めた最小公倍数を$\lambda$(n)とする。導いた$\lambda$(n)を用いて公開鍵の1つであるeを求める。eが0以上$\lambda$(n)未満かつ、$\lambda$(n)とeの最小公約数が1となるようにeを定める。

\begin{align*}
  \lambda(n)&= LCM(p - 1, q- 1)\\
  p - 1&= 18 = <pfac18>\\
  q - 1&= 30 = <pfac30>\\
  \lambda(n)&= <pfaclast> = 90\\\\
  GCD(e, \lambda(n))&= 1\\
  <gcdlist>
\end{align*}
$\lambda(n)$: <ln>\\
eの候補: <es>\\
e: <e>

導いた$\lambda$(n)を用いて秘密鍵dを求める。dを求める式を以下に示す。
\begin{align*}
  d = \frac{1}{e} \bmod \{\lambda(n)\}
\end{align*}

式を変形して、$(d \times e - 1) \bmod \lambda(n) = 0$となるようにdを決定する。
\begin{align*}
  (d \times e - 1) \bmod \lambda(n)&= 0\\
  (<d> \times <e> - 1) \bmod <ln>&= 0\\
  <de> \bmod <ln> &=0
\end{align*}



\section{暗号化}
暗号文cを求める。暗号文は$c = m^e \bmod n$で求める。\\
公開鍵 e: <e>, n: <n>\\
平文 m: <m>

$m^e = <m>^{<e>}$
\begin{verbatim}
<hsk>
\end{verbatim}

$<me> \bmod n\\
= <me> \bmod <n>$
\begin{verbatim}
<hsd>
\end{verbatim}

暗号: <c>

よって、暗号文<c>が導かれた。また、復号の確認も行う。復号は、$m = c^d \bmod n$によって確認できる。\\
<ruizyo>
復号: <hukugo>

初めに求めた平文と同じ結果が得られたため、復号が正しく行われた。


% --- main content ---
\end{document}

