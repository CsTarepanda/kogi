\documentclass[uplatex]{jsarticle}

\usepackage{amsmath}

\title{情報セキュリティーRSA暗号}
\author{c0115114 菅野 路哉}
\date{2016年06月18日}

\allowdisplaybreaks
\usepackage[dvipdfmx]{graphicx}
\begin{document}
\maketitle
% --- main content ---

\section{平文}
自分の学籍番号を用いて平文を求める。

初めに、十分に大きい素数である任意のpとqを決定する。
今回の暗号化では、pとqは19, 31を用いる。

次に、pとqを掛けた値をnとし、学籍番号(115114)をn - 2で割った余りに2を足したものを平文mとする。また、nは公開鍵の1つである。
\begin{align*}
  p&=19\\
  q&=31\\
  n&=p \times q = 19 \times 31 = 589\\
  m&=(115114 \bmod (n - 2)) + 2 = (115114 \bmod 589) + 2 = 64
\end{align*}
n: 589\\
m: 64


\section{秘密鍵生成}
p - 1とq - 1の最小公倍数を求める。求めた最小公倍数を$\lambda$(n)とする。導いた$\lambda$(n)を用いて公開鍵の1つであるeを求める。eが0以上$\lambda$(n)未満かつ、$\lambda$(n)とeの最小公約数が1となるようにeを定める。

\begin{align*}
  \lambda(n)&= LCM(p - 1, q- 1)\\
  p - 1&= 18 = 2 \times 3 \times 3\\
  q - 1&= 30 = 2 \times 3 \times 5\\
  \lambda(n)&= 2 \times 3 \times 5 \times 3 = 90\\\\
  GCD(e, \lambda(n))&= 1\\
  GCD(0, \lambda(n))&= 90\\
GCD(1, \lambda(n))&= 1\\
GCD(2, \lambda(n))&= 2\\
GCD(3, \lambda(n))&= 3\\
GCD(4, \lambda(n))&= 2\\
GCD(5, \lambda(n))&= 5\\
GCD(6, \lambda(n))&= 6\\
GCD(7, \lambda(n))&= 1\\
GCD(8, \lambda(n))&= 2\\
GCD(9, \lambda(n))&= 9\\
GCD(10, \lambda(n))&= 10\\
GCD(11, \lambda(n))&= 1\\
GCD(12, \lambda(n))&= 6\\
GCD(13, \lambda(n))&= 1\\
GCD(14, \lambda(n))&= 2\\
GCD(15, \lambda(n))&= 15\\
GCD(16, \lambda(n))&= 2\\
GCD(17, \lambda(n))&= 1\\
GCD(18, \lambda(n))&= 18\\
GCD(19, \lambda(n))&= 1\\
GCD(20, \lambda(n))&= 10\\
GCD(21, \lambda(n))&= 3\\
GCD(22, \lambda(n))&= 2\\
GCD(23, \lambda(n))&= 1\\
GCD(24, \lambda(n))&= 6\\
GCD(25, \lambda(n))&= 5\\
GCD(26, \lambda(n))&= 2\\
GCD(27, \lambda(n))&= 9\\
GCD(28, \lambda(n))&= 2\\
GCD(29, \lambda(n))&= 1\\
GCD(30, \lambda(n))&= 30\\
GCD(31, \lambda(n))&= 1\\
GCD(32, \lambda(n))&= 2\\
GCD(33, \lambda(n))&= 3\\
GCD(34, \lambda(n))&= 2\\
GCD(35, \lambda(n))&= 5\\
GCD(36, \lambda(n))&= 18\\
GCD(37, \lambda(n))&= 1\\
GCD(38, \lambda(n))&= 2\\
GCD(39, \lambda(n))&= 3\\
GCD(40, \lambda(n))&= 10\\
GCD(41, \lambda(n))&= 1\\
GCD(42, \lambda(n))&= 6\\
GCD(43, \lambda(n))&= 1\\
GCD(44, \lambda(n))&= 2\\
GCD(45, \lambda(n))&= 45\\
GCD(46, \lambda(n))&= 2\\
GCD(47, \lambda(n))&= 1\\
GCD(48, \lambda(n))&= 6\\
GCD(49, \lambda(n))&= 1\\
GCD(50, \lambda(n))&= 10\\
GCD(51, \lambda(n))&= 3\\
GCD(52, \lambda(n))&= 2\\
GCD(53, \lambda(n))&= 1\\
GCD(54, \lambda(n))&= 18\\
GCD(55, \lambda(n))&= 5\\
GCD(56, \lambda(n))&= 2\\
GCD(57, \lambda(n))&= 3\\
GCD(58, \lambda(n))&= 2\\
GCD(59, \lambda(n))&= 1\\
GCD(60, \lambda(n))&= 30\\
GCD(61, \lambda(n))&= 1\\
GCD(62, \lambda(n))&= 2\\
GCD(63, \lambda(n))&= 9\\
GCD(64, \lambda(n))&= 2\\
GCD(65, \lambda(n))&= 5\\
GCD(66, \lambda(n))&= 6\\
GCD(67, \lambda(n))&= 1\\
GCD(68, \lambda(n))&= 2\\
GCD(69, \lambda(n))&= 3\\
GCD(70, \lambda(n))&= 10\\
GCD(71, \lambda(n))&= 1\\
GCD(72, \lambda(n))&= 18\\
GCD(73, \lambda(n))&= 1\\
GCD(74, \lambda(n))&= 2\\
GCD(75, \lambda(n))&= 15\\
GCD(76, \lambda(n))&= 2\\
GCD(77, \lambda(n))&= 1\\
GCD(78, \lambda(n))&= 6\\
GCD(79, \lambda(n))&= 1\\
GCD(80, \lambda(n))&= 10\\
GCD(81, \lambda(n))&= 9\\
GCD(82, \lambda(n))&= 2\\
GCD(83, \lambda(n))&= 1\\
GCD(84, \lambda(n))&= 6\\
GCD(85, \lambda(n))&= 5\\
GCD(86, \lambda(n))&= 2\\
GCD(87, \lambda(n))&= 3\\
GCD(88, \lambda(n))&= 2\\
GCD(89, \lambda(n))&= 1\\
GCD(90, \lambda(n))&= 90
\end{align*}
$\lambda(n)$: 90\\
eの候補: 1, 7, 11, 13, 17, 19, 23, 29, 31, 37, 41, 43, 47, 49, 53, 59, 61, 67, 71, 73, 77, 79, 83, 89\\
e: 17

導いた$\lambda$(n)を用いて秘密鍵dを求める。dを求める式を以下に示す。
\begin{align*}
  d = \frac{1}{e} \bmod \{\lambda(n)\}
\end{align*}

式を変形して、$(d \times e - 1) \bmod \lambda(n) = 0$となるようにdを決定する。
\begin{align*}
  (d \times e - 1) \bmod \lambda(n)&= 0\\
  (53 \times 17 - 1) \bmod 90&= 0\\
  900 \bmod 90 &=0
\end{align*}
d: 53


\section{暗号化}
暗号文cを求める。暗号文は$c = m^e \bmod n$で求める。\\
公開鍵 e: 17, n: 589\\
平文 m: 64

$m^e = 64^{17}$
\begin{verbatim}
    64
*   64
------
   256
  384
------
  4096

    4096
*     64
--------
   16384
  24576
--------
  262144

    262144
*       64
----------
   1048576
  1572864
----------
  16777216

    16777216
*         64
------------
    67108864
  100663296
------------
  1073741824

   1073741824
*          64
-------------
   4294967296
  6442450944
-------------
  68719476736

    68719476736
*            64
---------------
   274877906944
  412316860416
---------------
  4398046511104

    4398046511104
*              64
-----------------
   17592186044416
  26388279066624
-----------------
  281474976710656

    281474976710656
*                64
-------------------
   1125899906842624
  1688849860263936
-------------------
  18014398509481984

    18014398509481984
*                  64
---------------------
    72057594037927936
  108086391056891904
---------------------
  1152921504606846976

   1152921504606846976
*                   64
----------------------
   4611686018427387904
  6917529027641081856
----------------------
  73786976294838206464

    73786976294838206464
*                     64
------------------------
   295147905179352825856
  442721857769029238784
------------------------
  4722366482869645213696

    4722366482869645213696
*                       64
--------------------------
   18889465931478580854784
  28334198897217871282176
--------------------------
  302231454903657293676544

    302231454903657293676544
*                         64
----------------------------
   1208925819614629174706176
  1813388729421943762059264
----------------------------
  19342813113834066795298816

    19342813113834066795298816
*                           64
------------------------------
    77371252455336267181195264
  116056878683004400771792896
------------------------------
  1237940039285380274899124224

   1237940039285380274899124224
*                            64
-------------------------------
   4951760157141521099596496896
  7427640235712281649394745344
-------------------------------
  79228162514264337593543950336

    79228162514264337593543950336
*                              64
---------------------------------
   316912650057057350374175801344
  475368975085586025561263702016
---------------------------------
  5070602400912917605986812821504


\end{verbatim}

$5070602400912917605986812821504 \bmod n\\
= 5070602400912917605986812821504 \bmod 589$
\begin{verbatim}
       8608832599173034984697475078
   --------------------------------
589)5070602400912917605986812821504
    4712
    ----
     358
     3586
     3534
     ----
       52
       520
         0
       ---
       520
       5202
       4712
       ----
        490
        4904
        4712
        ----
         192
         1920
         1767
         ----
          153
          1530
          1178
          ----
           352
           3529
           2945
           ----
            584
            5841
            5301
            ----
             540
             5402
             5301
             ----
              101
              1019
               589
              ----
               430
               4301
               4123
               ----
                178
                1787
                1767
                ----
                  20
                  206
                    0
                  ---
                  206
                  2060
                  1767
                  ----
                   293
                   2935
                   2356
                   ----
                    579
                    5799
                    5301
                    ----
                     498
                     4988
                     4712
                     ----
                      276
                      2766
                      2356
                      ----
                       410
                       4108
                       3534
                       ----
                        574
                        5741
                        5301
                        ----
                         440
                         4402
                         4123
                         ----
                          279
                          2798
                          2356
                          ----
                           442
                           4422
                           4123
                           ----
                            299
                            2991
                            2945
                            ----
                              46
                              465
                                0
                              ---
                              465
                              4650
                              4123
                              ----
                               527
                               5274
                               4712
                               ----
                                562

\end{verbatim}

暗号: 562

よって、暗号文562が導かれた。また、復号の確認も行う。復号は、$m = c^d \bmod n$によって確認できる。\\
$562^{1}$ = 562\\
$562^{2}$ = 315844\\
$562^{3}$ = 177504328\\
$562^{4}$ = 99757432336\\
$562^{5}$ = 56063676972832\\
$562^{6}$ = 31507786458731584\\
$562^{7}$ = 17707375989807150208\\
$562^{8}$ = 9951545306271618416896\\
$562^{9}$ = 5592768462124649550295552\\
$562^{10}$ = 3143135875714053047266100224\\
$562^{11}$ = 1766442362151297812563548325888\\
$562^{12}$ = 992740607529029370660714159149056\\
$562^{13}$ = 557920221431314506311321357441769472\\
$562^{14}$ = 313551164444398752546962602882274443264\\
$562^{15}$ = 176215754417752098931392982819838237114368\\
$562^{16}$ = 99033253982776679599442856344749089258274816\\
$562^{17}$ = 55656688738320493934886885265748988163150446592\\
$562^{18}$ = 31279059070936117591406429519350931347690550984704\\
$562^{19}$ = 17578831197866098086370413389875223417402089653403648\\
$562^{20}$ = 9879303133200747124540172325109875560579974385212850176\\
$562^{21}$ = 5552168360858819883991576846711750065045945604489621798912\\
$562^{22}$ = 3120318618802656774803266187852003536555821429723167450988544\\
$562^{23}$ = 1753619063767093107439435597572825987544371643504420107455561728\\
$562^{24}$ = 985533913837106326380962805835928204999936863649484100390025691136\\
$562^{25}$ = 553870059576453755426101096879791651209964517371010064419194438418432\\
$562^{26}$ = 311274973481967010549468816446442907980000058762507656203587274391158784\\
$562^{27}$ = 174936535096865459928801474842900914284760033024529302786416048207831236608\\
$562^{28}$ = 98314332724438388479986428861710313828035138559785468165965819092801154973696\\
$562^{29}$ = 55252654991134374325752373020281196371355747870599433109272790330154249095217152\\
$562^{30}$ = 31051992105017518371072833637398032360701930303276881407411308165546687991512039424\\
$562^{31}$ = 1745121956301984532454293250421769418671448483044160735096515518903723865122976615628\\8\\
$562^{32}$ = 9807585394417153072393128067370344132933540474708183331242417216238928121991128579833\\856\\
$562^{33}$ = 5511862991662440026684937973862133402708649746785999032158238475526277604559014261866\\627072\\
$562^{34}$ = 3097667001314291294996935141310518972322261157693731456072930023245768013762166015169\\044414464\\
$562^{35}$ = 1740888854738631707788277549416511662445110770623877078312986673064121623734337300525\\002960928768\\
$562^{36}$ = 9783795363631110197770119827720795542941522530906189180118985102620363525386975628950\\51664041967616\\
$562^{37}$ = 5498492994360683931146807343179087095133135662369278319226869627672644301267480303470\\19035191585800192\\
$562^{38}$ = 3090153062830704369304505726866646947464822242251534415405500730752026097312323930550\\24697777671219707904\\
$562^{39}$ = 1736666021310855855549132218499055584475230100145362341457891410682638666689526048969\\23880151051225475842048\\
$562^{40}$ = 9760063039767009908186123067964692384750793162816936358993349728036429306795136395207\\1220644890788717423230976\\
$562^{41}$ = 5485155428349059568400601164196157120229945757503118233754262547156473270418866654106\\4026002428623259191855808512\\
$562^{42}$ = 3082657350732171477441137854278240301569229515716752447369895551501937977975403059607\\7982613364886271665822964383744\\
$562^{43}$ = 1732453431111480370321919474104371049481906987832814875421881299944089143622176519499\\5826228711066084676192505983664128\\
$562^{44}$ = 9736388282846519681209187444466565298088317271620419599870972905685780987156632039587\\654340535619139588020188362819239936\\
$562^{45}$ = 5471850214959744060839563343790209697525634306650675815127486772995408914782027206248\\261739381017956448467345859904412844032\\
$562^{46}$ = 3075179820807376162191834599210097850009406480337679808101647566423419810107499289911\\523097532132091524038648373266280018345984\\
$562^{47}$ = 1728251059293745403151811044756074991705286441949776052153125932329961933280414600930\\275980813058235436509720385775649370310443008\\
$562^{48}$ = 9712770953230849165713178071529141453383709803757741413100567739694386065035930057228\\15101216938728315318462856805914946114468970496\\
$562^{49}$ = 5458577275715737231130806076199377496801644909711850674162519069708244968550192692162\\22086883919565313208976125524924199716331561418752\\
$562^{50}$ = 3067720428952244323895513014824050153202524439258060078879335717176033672325208292995\\16812828762795706023444582545007400240578337517338624\\
$562^{51}$ = 1724058881071161310029278314331116186099818734863029764330186673052930923846767060663\\28448809764691186785175855390294158935205025684744306688\\
$562^{52}$ = 9689210911619926562364544126540872965880981289930227275535649102557471792018830880927\\6588231087756446973268830729345317321585224434826300358656\\
$562^{53}$ = 5445336532330398728048873799115970606825111484940787728851034795637299147114582955081\\3442585871319123198977082869892068334730896132372380801564672\\
$54453365323303987280488737991159706068251114849407877288510347956372991471145829550813442585871319\\123198977082869892068334730896132372380801564672 \bmod 589 = 64$

復号: 64

初めに求めた平文と同じ結果が得られたため、復号が正しく行われた。


% --- main content ---
\end{document}


